\documentclass[a4]{article}

\usepackage[left=2cm,right=2cm,top=2cm,bottom=2cm]{geometry} 

\usepackage[utf8]{inputenc}   % otra alternativa para los caracteres acentuados y la "ñ"
\usepackage[           spanish % para poder usar el español
                      ,es-tabla % para los captions de las tablas
                       ]{babel}   
\decimalpoint %para usar el punto decimal en vez de coma para los números con decimales

%\usepackage{beton}
%\usepackage[T1]{fontenc}

\usepackage{parskip}
\usepackage{xcolor}

\usepackage{caption}

\usepackage{enumerate} % paquete para poder personalizar fácilmente la apariencia de las listas enumerativas

\usepackage{graphicx} % figuras
\usepackage{subfigure} % subfiguras

\usepackage{amsfonts}
\usepackage{amsmath}

\definecolor{gris}{RGB}{220,220,220}
	
\usepackage{float} % para controlar la situación de los entornos flotantes

\restylefloat{figure}
\restylefloat{table} 
\setlength{\parindent}{0mm}


\usepackage[bookmarks=true,
            bookmarksnumbered=false, % true means bookmarks in 
                                     % left window are numbered
            bookmarksopen=false,     % true means only level 1
                                     % are displayed.
            colorlinks=true,
            allcolors=blue,
            urlcolor=cyan]{hyperref}
\definecolor{webblue}{rgb}{0, 0, 0.5}  % less intense blue


\title{Aprendizaje Automático: Proyecto Final \\ Devanagari Handwritten Characters}

\author{David Cabezas Berrido y Patricia Córdoba Hidalgo}

\date{}

\begin{document}

\maketitle
\tableofcontents

\newpage

\section{Problema}

El problema a resolver consiste en clasificar caracteres de la
escritura Devanagari. Nuestros datos son imágenes de estos símbolos
escritos a mano, cada uno de ellos con una etiqueta especificando el
símbolo que representa la imagen.

Inicialemente, nuestro espacio de características $\mathcal{X}$ está
formado por imágenes de $32 \times 32$ píxeles cada una, con un marco
de $2$ píxeles por cada uno de los $4$ lados. El conjunto de
etiquetas, $Y$ son los $46$ caracteres que consideramos, $36$ letras y
$10$ dígitos (del $0$ a $9$). La función objetivo $f$ que buscamos
aproximar es aquella que a una cuadrícula de píxeles representando un
símbolo Devanagari manuscrito le haga corresponder la clase del
símbolo correspondiente.

\section{Dataset. Conjuntos de train y test}

El conjunto de datos de los que disponemos \\
\href{https://archive.ics.uci.edu/ml/datasets/Devanagari+Handwritten+Character+Dataset}{https://archive.ics.uci.edu/ml/datasets/Devanagari+Handwritten+Character+Dataset}
consta de 92000 instancias (imágenes etiquetadas), 2000 de cada una de
las 46 clases. Los datos vienen divididos en train: 78200 instancias
(85\%), 1700 por clase; y test: 13800 instancias (15\%), 300 por
clase.

Existen algunos artículos y proyectos relativos a este dataset, por lo
que mantener esta división entre train y test nos permitirá
comparar los resultados que logren nuestros modelos y procesamiento
con los resultados de otros diseñados por terceros.
También sabemos que no existen datos perdidos y las clases están
perfectamente balanceadas.

En la descripción del dataset se informa de que estos datos proceden
de documentos escritos, pero desconocemos su procedencia y sus
autores. En problemas de reconocimiento de símbolos manuscritos existe
la dificultad de que un modelo aprenda a reconocer símbolos de un
único autor o un conjunto de autores, lo que conlleva sobreajuste y a
estimaciones por validación poco fiables, ya que la muestra de
entrenamiento no termina de ser representativa y el modelo se adapta a
ese sesgo. En nuestro caso, desconocemos si los datos de train y test
corresponden a símbolos trazados por distintas personas o no, con lo
que tenemos otra razón más para respetar la partición de conjuntos de
train y test existente.

% Extraemos un conjunto de validación del 30\% de los datos de
% entrenamiento para consultar la eficacia práctica de ciertas
% decisiones que tomamos durante el preprocesamiento, así como para
% estimar los hiperparámetros usados en cada modelo. Esto nos deja con
% 54740 (70\%) datos para entrenar cada alternativa y 23460 datos para
% validar. Debido a la abundancia de datos, podemos esperar que los
% scores obtenidos al validar sean representativos de la bondad real de
% un modelo o procesamiento, así que decidimos no usar validación
% cruzada, ya que es computacionalmente muy costosa y el ajuste de los
% modelos es lento debido a la cantidad de datos.

% Para ello usamos la función \texttt{train\_test\_split} de
% \textit{sklearn} e indicamos mediante la opción \texttt{stratify} que
% queremos preservar la proporción de elementos de cada clase en cada
% uno de los conjuntos, de forma que sigan perfectamente balanceadas.

\subsection{Formato de los datos}

En el fichero \texttt{png\_to\_np.py}, guardamos las imágenes,
originalmente en formato \texttt{png}, como array. Para ello leemos
cada imagen como array de escala de grises usando la función
\texttt{imread} de \textit{matplotlib} y eliminamos los dos píxeles de
marco por cada lado, quedándonos con matrices de $28 \times 28$ de
valores flotantes entre $0$ y $1$ representando la intensidad de gris
en cada pixel. El resultado es guardado en disco, lo hacemos con la
función \texttt{saveGrey}, que usa la función
\texttt{savez\_compressed} de \textit{numpy} para almacenarlo en
formato \texttt{npz}.

Este fichero solo se ejecuta una vez para cambiar el formato de los
datos. Tras esto, podemos usar los datos guardados en disco para las
sucesivas ejecuciones del código. Proporcionamos los datos ya
transformados, aunque en el fichero \texttt{main.py} se puede alterar
la variable \texttt{PNG\_TO\_NP} para ejecutar el script que carga las
imágenes y almacena estos datos en disco.

\section{Preprocesamiento}

% Comentar justificaciones: 3 modelos ajustados con parámetros por defecto. WIDTH y BLOCK_REDUCE

Preprocesamos los datos usando las funciones implementadas en el
fichero \texttt{preprocessing.py}. El preprocesamiento se realiza
imagen a imagen, siendo el resultado sólo dependiente de la propia
imagen y por tanto paralelizable. Para cada imagen realizamos dos operaciones: centrado y reescalado, y downsampling. Tras este proceso, nos quedan $144$ características ($12 \times 12$) y comprobamos con la función \texttt{VarianceThreshold} de \textit{sklearn} que no hay características con varianza $0$ en el conjunto de train (todas aportan algo de información). Consideramos que no es necesario normalizar las variables, ya que todas tienen la misma naturaleza (intensidad de gris en un píxel) y la misma escala (entre $0$ y $1$).

\subsection{Centrado y reescalado}

Esta operación la realiza la función \texttt{centerAndResize}. Ante
una imagen, calculamos un umbral con la ayuda del \href{https://en.wikipedia.org/wiki/Otsu%27s_method}{método de Otsu}
  para
\href{https://en.wikipedia.org/wiki/Thresholding_(image_processing)}{thresholding} (usando la función \texttt{threshold\_otsu} de la librería \textit{skimage}). 

Para calcular la caja englobante del carácter, usamos el
\href{https://en.wikipedia.org/wiki/Closing_%28morphology%29}{closing}
  (función \texttt{closing} de \textit{skimage}) de la imagen
  resultante de considerar los píxeles con intensidad superior a este
  umbral. Recortamos el exterior de la caja y reescalamos la imagen a
  \texttt{WIDTH $\times$ WIDTH} para trabajar con un tamaño de imagen
  unificado (función \texttt{resize} de \textit{skimage}).

%COMENTAR COMO ELEGIMOS WIDTH, ya que muchas no se recortan

\begin{figure}[H]
  \centering
  \subfigure[Muestra antes del preprocesado]{\includegraphics[width=87mm]{imgs/sample-kna.png}}
  \subfigure[Thresholding y closing]{\includegraphics[width=87mm]{imgs/closing-thresholding.png}}
  \subfigure[Recortado de la caja englobante]{\includegraphics[width=87mm]{imgs/crop.png}}
  \subfigure[Reescalado]{\includegraphics[width=87mm]{imgs/resize.png}}
  \caption{Proceso de centrado y reescalado sobre una instancia correspondiente al carácter kna}
  \label{fig:centerAndResize}
\end{figure}

\subsection{Downsampling}

Para reducir la dimensionalidad, realizamos un downsampling o
reducción por bloques de la imagen, agrupando cada bloque de 4
píxeles en uno usando la media de sus valores. Esta reducción se lleva
a cabo con la función \texttt{block\_reduce} de \textit{skimage}.

% Comentar que esta decisión es arriesgada y que nos hemos basado en
% validación con los 3 modelos ajustados

\begin{figure}[H]
  \centering
  \includegraphics[width=87mm]{imgs/downsampling.png}
  \caption{Resultado del preprocesamiento (tras combinar centrado y reescalado con downsampling)}
  \label{fig:downsampling}
\end{figure}

\section{Random Forest}

El modelo de random forest construye \texttt{n\_estimators} árboles de decisión usando la totalidad de la muestra para la construcción de cada uno, pero sólo un subconjunto de las características para disminuir la correlación entre los árboles. Usamos la raíz cuadrada del número de características, que es un valor adecuado según lo estudiado en teoría y además, el valor que recomienda la implementación de \textit{sklearn}. Tras esto, hace una media de dichos árboles para controlar el overfitting y reducir la variabilidad.

Elegimos este modelo por su capacidad para conseguir un bajo sesgo combinada con una baja variabilidad. Además, los árboles de decisión son muy adecuados para clasificación cuando el número de clases es elevado (basta asignar una clase a cada nodo), y en nuestro caso tenemos 46 clases.

\subsection{Función de pérdida}

La función de pérdida que intenta minimizar cada uno de los árboles de decisión que promedia random forest es:
\[R(T) = \sum\limits_{m=1}^{|T|} N_mQ_m(T)\]
donde $|T|$ es el número de nodos del árbol $T$, $N_m$ es el número de instancias que caen en el nodo terminal $m$ y $Q_m(T)$ la medida de impureza del nodo terminal $m$. Como medida de impureza, tomamos Gini Index, aunque la entropía cruzada proporciona resultados parecidos.

\subsection{Estimación de hiperparámetros}

De los múltiples hiperparámetros que podríamos ajustar (profundidad máxima de cada árbol, máximo número de nodos terminales, mínimo número de muestras en cada nodo, \ldots), sólo buscaremos un valor adecuado para el número de estimadores a tener en cuenta, manteniendo el resto por defecto.

Para elegir un valor de \texttt{n\_estimators}, representamos la
accuracy media obtenida usando validación cruzada con tres
subdivisiones (hacer la media de tres ejecuciones le da cierta
estabilidad a los resultados) según diferentes valores de éste
parámetro entre $50$ y $300$. Observamos que la accuracy es
creciente con el múmero de estimadores, pero nos preguntamos si merece
la pena ese crecimiento a costa del incremento del tiempo de ejecución
que conlleva el uso de un mayor número de estimadores. Por eso,
tomamos nuevas mediciones entre $200$ y $300$, donde vimos que este
crecimiento parece saturar a partir de $275$ árboles. Así, decidimos
que el valor óptimo estaría dentro de este intervalo y repetimos ahí
el experimento, obteniendo la máxima accuracy con $290$
estimadores. Puesto que se hizo de $5$ en $5$, para obtener un valor
más exacto, tomamos mediciones cada $2$ estimadores entre $285$ y
$295$. En la cuarta gráfica, observamos dos picos, en $287$ y
$293$. Viendo la escala del eje, concluimos que la accuracy
satura en este tramo y no compensa seguir aumentando el número de
estimadores, por lo que elegimos $287$ árboles (el primer pico) como valor para el hiperparámetro.

\begin{figure}[H]
  \centering
  \subfigure[Entre $50$ y $300$, de $50$ en $50$]{\includegraphics[width=87mm]{imgs/grafNestimators1}}
  \subfigure[Entre $200$ y $300$, de $25$ en $25$]{\includegraphics[width=87mm]{imgs/grafNestimators2}}
  \subfigure[Entre $275$ y $300$, de $5$ en $5$]{\includegraphics[width=87mm]{imgs/grafNestimators3}}
  \subfigure[Entre $285$ y $295$, de $2$ en $2$]{\includegraphics[width=87mm]{imgs/grafNestimators4}}
  \caption{Accuracy media de tres validaciones para distintos valores del hiperparámetro \texttt{n\_estimators}}
  \label{fig:HyperparametersRF}
\end{figure}

A continuación medimos la accuracy sobre el conjunto de train,
obteniendo $1$. Esto nos hace pensar que el modelo sobreajusta e
intentamos regularizar para reducir la complejidad (el número de
nodos) de cada árbol penalizando con el parámetro $\alpha$. Quedando
la función a minimizar como:
\[R_\alpha(T)=R(T)+\alpha|T|\] donde $|T|$ es el número de nodos
terminales del árbol $T$ y $R$ es la función de pérdida asociada a la impureza de los nodos terminales ($Q$) promediados por el número de muestras que instancias que caen cada nodo ($N$).

Originalmente el valor por defecto de $\alpha$ es 0, hemos probado a
incrementarlo y medir la accuracy media de tres subdivisiones de validación cruzada para cada uno de los diferentes valores de $\alpha$ probados. Claramente, la accuracy decrece al aumentar alpha. Observando la escala del eje en la última gráfica, concluimos que no mejoramos nada aumentando éste, por lo que mantenemos el valor inicial de $\alpha$, $0$.

\begin{figure}[H]
  \centering
  \subfigure[Entre $0$ y $0.0001$, $6$ mediciones]{\includegraphics[width=57mm]{imgs/grafAlpha1}}
  \subfigure[Entre $0$ y $0.00005$, $11$ mediciones]{\includegraphics[width=57mm]{imgs/grafAlpha2}}
  \subfigure[Entre $0$ y $0.00001$, $6$ mediciones ]{\includegraphics[width=57mm]{imgs/grafAlpha3}}
  \caption{Accuracy media de tres validaciones para distintos valores del hiperparámetro $\alpha$}
  \label{fig:HyperparametersRF}
\end{figure}

\section{Multilayer Perceptron (MLP)}

El modelo de perceptrón multicapa usado consta de tres capas: la capa de entrada, dos ocultas y una de salida. Como función de activación en cada neurona hemos considerado $\tanh$ y como algoritmo para ajustar los pesos usamos Adam, como nos recomendaron en teoría y como recomienda la documentación de \textit{sklearn} para datasets grandes como es el nuestro.

Al igual que random forest, el perceptrón multicapa tiene suficiente complejidad para obtener un bajo sesgo y facilidad para clasificación no binaria. Para afrontar el problema del sobreajuste, podemos utilizar early stopping como regularización, ya que disponemos de un número elevado de muestras de entrenamiento y podemos permitirnos sacrificar algunas con el fin de combatir el sobreajuste.

Por tanto, consideramos este modelo adecuado para el problema.

\subsection{Estimación de hiperparámetros}

Estimamos el número de neuronas, \texttt{N\_NEUR}, que tiene cada una de las capas ocultas. Como nos recomendaban valores entre $50$ y $100$ hicimos las mediciones de la accuracy media usando validación cruzada con dos subdivisiones del modelo con \texttt{N\_NEUR} en dicho rango. Observamos que el máximo se alcanza en torno a $60$ neuronas por capa, luego repetimos las mediciones ahora en el intervalo $[55,70]$. A pesar de que la diferencia entre la accuracy es leve, seguimos buscando un valor para \texttt{N\_NEUR} en torno a $60$, que es donde conseguimos la accuracy más alta. Al medir la accuracy en el intervalo $[58,62]$, obtenemos el máximo en $59$. Por tanto, cada capa oculta del MLP usado para el ajuste tendrá 59 neuronas.

\begin{figure}[H]
  \centering
  \subfigure[Entre $50$ y $100$, de $10$ en $10$]{\includegraphics[width=57mm]{imgs/grafNneur1}}
  \subfigure[Entre $55$ y $70$, de $5$ en $5$]{\includegraphics[width=57mm]{imgs/grafNneur2}}
  \subfigure[Entre $58$ y $62$, de $1$ en $1$ ]{\includegraphics[width=57mm]{imgs/grafNneur3}}
  \caption{Accuracy media de dos validaciones para distintos valores del hiperparámetro \texttt{N\_NEUR}}
  \label{fig:HyperparametersRF}
\end{figure}

\end{document}
