\documentclass[a4]{article}

\usepackage[left=3cm,right=3cm,top=2cm,bottom=2cm]{geometry} 

\usepackage[utf8]{inputenc}   % otra alternativa para los caracteres acentuados y la "ñ"
\usepackage[           spanish % para poder usar el español
                      ,es-tabla % para los captions de las tablas
                       ]{babel}   
\decimalpoint %para usar el punto decimal en vez de coma para los números con decimales

\usepackage[T1]{fontenc}
\usepackage{lmodern}

\usepackage{parskip}
\usepackage{xcolor}

\usepackage{caption}

\usepackage{enumerate} % paquete para poder personalizar fácilmente la apariencia de las listas enumerativas

\usepackage{graphicx} % figuras
\usepackage{subfigure} % subfiguras

\usepackage{amsfonts}
\usepackage{amsmath}

\definecolor{gris}{RGB}{220,220,220}
	
\usepackage{float} % para controlar la situación de los entornos flotantes

\restylefloat{figure}
\restylefloat{table} 

\newcommand{\HRule}{\rule{\linewidth}{0.5mm}}

\author{David Cabezas Berrido}
\date{\vspace{-5mm}}

\title{\huge Práctica 1 \HRule\vspace{-4mm}}

\begin{document}
\maketitle
\tableofcontents

\section{Ejercicio sobre la búsqueda iterativa de óptimos: \\ Gradiente descendiente}

\subsection{Minimizar la función $E(u,v)$}

La función a minimizar es $E(u,v)=(ue^v-2ve^{-u})^2$, le aplicamos el algoritmo del gradiente 
descendente partiendo del punto $w=(1,1)$ con tasa de aprendizaje $\eta=0.1$.
La función es no negativa y sabemos que si encontramos un cero será un mínimo absoluto,
aceptamos un margen de error $\varepsilon=10^{-14}$ y nos interesa el punto en el que se alcanza
y las iteraciones necesarias para alcanzarlo. También he fijado un máximo de 100 iteraciones,
ya que no tengo asegurado encontrar un 0 y hay que añadir esa condición para asegurarnos
de que va a parar en algún momento.

He usado la librería \texttt{sympy} para el cálculo de las derivadas parciales en el programa.
También las he calculado analíticamente y el gradiente queda de este modo:

\begin{equation*}
\nabla E(u,v)=
\begin{pmatrix}
\frac{\partial E}{\partial u}(u,v) \vspace{2mm}\\
\frac{\partial E}{\partial v}(u,v)
\end{pmatrix}=
\begin{pmatrix}
    2(ue^v-2ve^{-u})(e^v+2ve^{-u}) \vspace{2mm}\\
    2(ue^v-2ve^{-u})(ue^v-2e^{-u})
    \end{pmatrix}
\end{equation*}

He ejecutado el algoritmo y, en sólo 10 iteraciones, he encontrado un valor por debajo de $\varepsilon$,
en el punto $w=(0.0447362903977822,0.0239587140991418)$.

\subsection{Estudiar la dependencia del \textbf{learning rate} ($\eta$)}

He utilizado el algoritmo del gradiente descendente para buscar un mínimo local de la función
$f(x,y)=(x-2)^2+2(y+2)^2+2\sin(2\pi x)\sin(2\pi y)$ partiendo desde el punto $(1,-1)$. Esta vez
realiza un número de iteraciones fijo (50 iteraciones), ya que desconozco los valores que puede
tomar la función y no puedo incorporar un criterio de parada como el de antes. Sí que podría incluir
como criterio la norma del gradiente, ya que ésta es casi 0 cuando estamos muy cerca del mínimo local.

El objetivo ésta vez es estudiar cómo afecta a la eficacia del algoritmo el valor escogido para la
tasa de aprendizaje $\eta$, para ello he ejecutado el algoritmo con dos valores diferentes de $\eta$
(0.01 y 0.1) y he medido el valor de la función en cada iteración, para estudiar la velocidad a la que
decrece o si se salta el mínimo y vuelve a crecer. En las siguientes gráficas se muestra cómo evoluciona
el valor de la función en cada iteración.

\begin{figure}[H]
    \centering    
    \subfigure[$\eta=0.01$]{\includegraphics[width=77mm]{imgs/gd_0,01.png}}
    \subfigure[$\eta=0.1$]{\includegraphics[width=77mm]{imgs/gd_0,1.png}}
    \caption{Comparación de la eficacia del algoritmo para dos valores de $\eta$.}
    \label{fig:comp-eta}
\end{figure}

Lo deseable es que la función decrezca lo más rápido posible. En el primer caso ($\eta=0.01$) se llega a un mínimo local
bastante rápido (se queda muy cerca en la quinta iteración) en el que la función toma el valor $-0.38124949743810027$,
en la segunda gráfica se aprecian valores de la función cercanos a -2, luego está claro que ese mínimo no es absoluto.

En cambio, la tasa de aprendizaje $\eta=0.1$ es demasiado alta y esto provoca que se salte el punto donde se alcanza el
mínimo local y el algoritmo no converja en el segundo caso. Es importante que la tasa de aprendizaje no sea demasiado alta
para evitar esto.
\end{document} 