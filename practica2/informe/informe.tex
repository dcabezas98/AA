\documentclass[a4]{article}

\usepackage[left=3cm,right=3cm,top=2cm,bottom=2cm]{geometry} 

\usepackage[utf8]{inputenc}   % otra alternativa para los caracteres acentuados y la "ñ"
\usepackage[           spanish % para poder usar el español
                      ,es-tabla % para los captions de las tablas
                       ]{babel}   
\decimalpoint %para usar el punto decimal en vez de coma para los números con decimales

\usepackage[T1]{fontenc}
\usepackage{lmodern}

\usepackage{parskip}
\usepackage{xcolor}

\usepackage{caption}

\usepackage{enumerate} % paquete para poder personalizar fácilmente la apariencia de las listas enumerativas

\usepackage{graphicx} % figuras
\usepackage{subfigure} % subfiguras

\usepackage{amsfonts}
\usepackage{amsmath}

\definecolor{gris}{RGB}{220,220,220}
	
\usepackage{float} % para controlar la situación de los entornos flotantes

\restylefloat{figure}
\restylefloat{table} 

\newcommand{\HRule}{\rule{\linewidth}{0.5mm}}

\author{David Cabezas Berrido}
\date{\vspace{-5mm}}

\title{\huge Aprendizaje Automático: Práctica 2 \HRule\vspace{-4mm}}

\begin{document}
\maketitle
\tableofcontents

\newpage

\section{Ejercicio sobre la complejidad de H y el ruido}

\subsection{Dibujar nubes de puntos}

Utilizando las funciones \texttt{simula\_unif} y \texttt{simula\_gaus},
he generado dos muestras en dimensión 2, de tamaño $N=50$. \\
La primera muestra es una uniforme en el cuadrado
$[-50,50]\times[-50,50]$ 
y la segunda muestra es una normal de media $(0,0)$ y varianza $(5,7)$ 
(desviación típica $(\sqrt{5},\sqrt{7})$). \\
He mantenido fija la escala de los ejes para que se aprecie que los puntos
generados por la normal están mucho más concentrados en torno al punto $(0,0)$,
la concentración es ligeramente mayor en el eje de abscisas 
debido a que tiene menor varianza, pero apenas se aprecia.

\vspace{-3mm}
\begin{figure}[H]
    \centering
    \subfigure[$U\big((-50,50)\times(-50,50)\big)$]{\includegraphics[width=77mm]{imgs/uniform.png}}
    \subfigure[$N\big((0,0),(\sqrt{5},\sqrt{7})\big)$]{\includegraphics[width=77mm]{imgs/normal.png}}
    \caption{Muestras uniforme y normal}
    \label{fig:uniform-normal}
\end{figure}
\vspace{-4mm}

\subsection{Muestra etiquetada por una recta}

He generado (con \texttt{simula\_unif}) una muestra de $N=100$ puntos
del cuadrado $[-50,50]\times[-50,50]$ y los he etiquetado con
el signo de la distancia a una recta generada por \texttt{simula\_recta} que corta a dicho cuadrado.
Esto quiere decir que si la recta es $y=ax+b$, la etiqueta
de un punto $(x,y)$ es el signo de la función $y-ax-b$ (+1 si el punto 
queda por encima de la recta y -1 si queda por debajo).
Obviamente la recta clasifica a la perfección todos los puntos.

A continuación, he añadido ruido a las etiquetas de la muestra
(10\% en cada clase). Claramente la recta tiene un 10\% de puntos mal clasificados ahora.

\begin{figure}[H]
    \centering
    \subfigure[Muestra sin ruido]{\includegraphics[width=77mm]{imgs/x-f.png}}
    \subfigure[Muestra con ruido]{\includegraphics[width=77mm]{imgs/x-f_noise.png}}
    \caption{Muestra y recta etiquetadora}
    \label{fig:muestra-ruido}
\end{figure}

\subsection{Comparación con otras funciones frontera}

Ahora he representado la misma muestra (con ruido) junto con otras funciones,
he calculado también el error que cometen dichas funciones funciones a
la hora de clasificar la muestra.

En la siguiente figura se ve que puntos clasifica bien y mal
cada función, los bien clasificados son los que están en la región de su color.
Incluyo también el porcentaje de puntos mal clasificados para cada función.

\vspace{-3mm}

\begin{figure}[H]
    \centering
    
    \subfigure[$f(x,y)=(x-10)^2+(y-20)^2-400$;\ \ Error: 58\%]{\includegraphics[width=77mm]{imgs/x-f1_noise.png}}
    \subfigure[$f(x,y)=0.5(x+10)^2+(y-20)^2-400$;\ \ Error: 68\%]{\includegraphics[width=77mm]{imgs/x-f2_noise.png}}
    
    \subfigure[$f(x,y)=0.5(x-10)^2-(y+20)^2-400$;\ \ Error: 35\%]{\includegraphics[width=77mm]{imgs/x-f3_noise.png}}
    \subfigure[$f(x,y)=y-20x^2-5x+3$;\ \ Error: 47\%]{\includegraphics[width=77mm]{imgs/x-f4_noise.png}}
    
    \caption{Muestra con otras funciones como delimitadores}
    \label{fig:muestra-otras}
\end{figure}

\vspace{-3mm}

Ninguna de estas cuatro funciones ha conseguido si quiera acercarse
al error del 10\% que presentaba la recta. Esto era de esperar,
ya que la recta la he usado para etiquetar la muestra (salvo ese 10\% de ruido)
y las funciones estaban prefijadas antes (no han recibido ningún aprendizaje ni ajuste).

Sin embargo, es perfectamente posible que el ruido de la muestra
provoque que el mejor ajuste no sea la misma recta que usé
para etiquetarla, en este caso se puede producir sobreajuste
si la clase H es lo suficientemente compleja como para ajustar el
ruido.

Por ejemplo, si ajustase la muestra sin ruido mediante
una parábola $y=ax^2+bx+c$ el peso $a$ obtendría un valor muy cercano a 0.
Pero en la muestra con ruido, esto puede no ocurrir, es probable
que el valor que se obtenga para $a$ al ajustar la muestra con
ruido no sea cercano a 0, lo que ajustaría mejor la muestra (con ruido)
pero se alejaría de la función objetivo, dando lugar a sobreajuste.

\end{document}